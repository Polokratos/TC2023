\documentclass{article}

\usepackage{polski}
\usepackage[utf8]{inputenc}
\usepackage{graphicx}
\usepackage{amsfonts}
\usepackage{amssymb}
\usepackage{amsmath}
\usepackage{listings}
\usepackage{breqn}

\author{Maciej Pieta \and Piotr Koproń \and Jakub Woś \and Rafał Piwowar}
\date{Marzec 2023}

\title{Technika Cyfrowa. \\ Ćwiczenie 1.}

\begin{document}
\maketitle
\newpage
\section{Zadanie 1a}
\subsection{Treść zadania}
\paragraph{}
Bazując wyłącznie na dwuwejściowych bramkach logicznych NAND, proszę od podstaw  zaprojektować, zbudować i przetestować układ realizujący funkcję logiczną:
\begin{equation}
Y = \overline{A} \text{ xor } (B + C)
\end{equation}
\subsection{Rozwiązanie teoretyczne}
Dokonujemy następujących przekształceń:
\begin{align*}
 Y &= \overline{A} \text{ xor } (B + C) \\
 &= \overline{A} \cdot \overline{(B+C)} + \overline{\overline{A}} \cdot (B+C) \\
&= \overline{A} \cdot (\overline{B}+\overline{C}) + A \cdot (B+C) \\
&= \overline{A} \cdot \overline{\overline{(\overline{B} \cdot \overline{C})}} + A \cdot \overline{\overline{(B+C)}} \\
&= \overline{A} \cdot \overline{\overline{(\overline{B} \cdot \overline{C})}} + A \cdot \overline{(\overline{B} \cdot \overline{C})} \\
|K = \overline{(\overline{B} \cdot \overline{C})}| &= \overline{A} \cdot \overline{K} + A \cdot K \\
&= \overline{\overline{\overline{A} \cdot \overline{K} + A \cdot K}} \\
&= \overline{\overline{\overline{A} \cdot \overline{K}} \cdot \overline{A \cdot K}} \\
&= \overline{\overline{\overline{A} \cdot \overline{\overline{(\overline{B} \cdot \overline{C})}}} \cdot \overline{A \cdot \overline{(\overline{B} \cdot \overline{C})}}}
\end{align*}
\subsection{Implementacja układu w programie Multisim}
\subsection{Wnioski}
\newpage
\section{Zadanie 1b}
\subsection{Treść zadania}
\paragraph{}
Rozważmy pomieszczenie w którym znajdują się: drzwi wejściowe i dwa okna (wszystko wyposażone w czujniki stanu zamknięcia). Poza tym znajduje się tam: czujnik ruchu, syrena alarmowa (może być reprezentowana wskaźnikiem LED), dwa przyciski: uzbrojenia i rozbrojenia alarmu, dwa wskaźniki LED: alarm uzbrojony i alarm wyłączony, LEDowy czerwony sygnalizator problemu załączenia alarmu.
\paragraph{}
Alarm można uzbroić dedykowanym przyciskiem tylko wtedy, gdy w pomieszczeniu nie wykryto ruchu, a drzwi i okna są skutecznie zamknięte. Wówczas powinna zaświecić się kontrolka uzbrojenia alarmu. Jeśli warunki te nie są spełnione, zaświeca się czerwony sygnalizator problemu, a alarm pozostaje rozbrojony, co ciągle wówczas sygnalizuje stosowny wskaźnik LED.
\paragraph{}
Poprawne uzbrojenie alarmu powoduje zgaszenie się wskaźnika rozbrojenia alarmu i sygnalizatora problemu (jeśli jest zaświecony) oraz powoduje zaświecenie się wskaźnika uzbrojenia alarmu.
\paragraph{}
Alarm uruchamia się, gdy system alarmowy jest uzbrojony i wykryty jest ruch lub sygnalizowane jest otwarcie: drzwi lub któregoś z okien.
\paragraph{}
W oparciu o dowolne bramki logiczne, przełączniki i wskaźniki LED, proszę zaprojektować, zminimalizować, zbudować i przetestować układ realizujący funkcję opisanego wyżej systemu alarmowego. Rolę czujników mogą tutaj pełnić dowolne (dostępne w Multisimie) źródła sygnału cyfrowego.
\subsection{Rozwiązanie teoretyczne}
\subsection{Implementacja układu w programie Multisim}
\subsection{Wnioski}
\end{document}