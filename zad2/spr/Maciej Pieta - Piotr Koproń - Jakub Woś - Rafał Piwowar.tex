\documentclass{article}

\usepackage{polski}
\usepackage[utf8]{inputenc}
\usepackage{graphicx}
\usepackage{amsfonts}
\usepackage{amssymb}
\usepackage{amsmath}
\usepackage{listings}
\usepackage{breqn}
\usepackage{float}

\author{Maciej Pieta \and Piotr Koproń \and Jakub Woś \and Rafał Piwowar}
\date{Marzec 2023}

\title{Technika Cyfrowa. \\ Ćwiczenie 2.}

\begin{document}
\maketitle
\newpage

\section{Zadanie 2a}
\paragraph{Treść zadania} Na podstawie dostępnych tabel prawdy, zaprojektować i praktycznie zrealizować synchroniczny przerzutnik D w oparciu o dostępny synchroniczny przerzutnik T, po czym proszę jednoznacznie przetestować poprawność jego działania w programie Multisim. 
\subsection{Ogólna idea rozwiązania}
\begin{figure}[H]
\includegraphics[width=0.7\textwidth]{DBB}
\end{figure}
Jako że realizacja ma opierać się o synchroniczny przerzutnik T, to schemat przyjmuje postać:
\begin{figure}[H]
\includegraphics[width=0.7\textwidth]{KBB}
\end{figure}
W celu wyznaczenia bramek logicznych zastosujemy następujący algorytm: \\
1. Wyznaczymy wzory przejścia dla przerzutników D oraz T. \\
2. Nadamy równoważność wzorom przejścia. \\
3. Otrzymamy zależność między sygnałami D,T, oraz Q. \\
4. Przekształcimy otrzymaną zależność do funkcji T od D i Q. \\
\newpage
\paragraph{Wzory przejścia}
Dla przerzutnika T: \\
\begin{tabular}{c c c}
T & Q & $Q_{T}^{+}$ \\
0 & 0 & 0 \\
0 & 1 & 1 \\
1 & 0 & 1 \\
1 & 1 & 0
\end{tabular}
$\implies$ Z definicji xor otrzymujemy $Q^{+}_{T} = T \text{ xor } Q$. (1) \\
Dla przerzutnika D: \\
\begin{tabular}{c c c}
D & Q & $Q_{D}^{+}$ \\
0 & 0 & 0 \\
0 & 1 & 0 \\
1 & 0 & 1 \\
1 & 1 & 1
\end{tabular}
$\implies $ Bezpośrednio otrzymujemy $Q_{D}^{+} = D$. (2)\\
Z (1) i (2), podstawiając $Q^{+}_{T} = Q^{+}_{D}$ otrzymujemy $D = T \text{ xor } Q$ (3).
\paragraph{Przekształcenie do funkcji}
Chcemy utworzyć funkcję T od D i Q, tak aby (3) zawsze było spełnione. Tworzymy tabelę, gdzie po lewej stronie będziemy mieć wartości niezależne, w środku - wyrażenie wymuszające, po prawej - wyrażenia zależne.
\begin{center}
\begin{tabular}{c c | c | c c }
D & Q & $D = T \text{ xor } Q$ & $T \text{ xor } Q$ & T \\
0 & 0 & 1 & 0 & 0\\
0 & 1 & 1 & 0 & 1\\
1 & 0 & 1 & 1 & 1\\
1 & 1 & 1 & 1 & 0
\end{tabular}
\end{center}
Usuwając kolumny '$D = T \text{ xor } Q$' i '$T \text{ xor } Q$' z powyższej tabeli, otrzymujemy:
\begin{tabular}{c c c }
D & Q & T \\
0 & 0 & 0\\
0 & 1 & 1\\
1 & 0 & 1\\
1 & 1 & 0
\end{tabular} $\implies$ Z definicji xor otrzymujemy $ T = D \text{ xor } Q$. \\
\newpage
\subsection{Implementacja w programie Multisim}
\begin{figure}[H]
\caption{Implementowany przerzutnik D.}
\includegraphics[width = \textwidth]{2aimp}
\end{figure}
\begin{figure}[H]
\caption{Układ SC1 powyższego rysunku.}
\includegraphics[width = \textwidth]{2asc1}
\end{figure}
\begin{figure}[H]
\caption{Układ testujący przerzutnik.}
\includegraphics[width = \textwidth]{2atest}
\end{figure}
\begin{figure}[H]
\caption{Ustawienia generatora słów (XWG1) i rezultat alalizatora logicznego (XLA1).}
\includegraphics[width = 0.4\textwidth]{2agen}
\includegraphics[width = 0.4\textwidth]{2aana}
\end{figure}
\newpage
\subsection{Wnioski}
\paragraph{Alternatywne koncepcje}
Rozważaliśmy podejście alternatywne, w którym zamiast przekształcać sygnał wejściowy, przekształcalibyśmy sygnał wyjściowy.
\begin{figure}[H]
\includegraphics[width = \textwidth]{2afail}
\end{figure}
 Szybko doszliśmy jednak do wniosku że takie podejście wymagałoby zastosowania drugiego przerzutnika, co mijałoby się z celem zadania, więc skupiliśmy się na opisanym wyżej podejściu.
\paragraph{Zastosowania}
Przerzutniki typu D mogą być stosowane na przykład w rejestrach, co zademonstrujemy w dalszej części sprawozdania - w którym zbudowaliśmy czterobitowy rejestr PISO w oparciu właśnie o przerzutniki D.
\section{Zadanie 2b}
\paragraph{Treść zadania}
Korzystając z wybranych przerzutników, proszę zbudować czterobitowy rejestr PISO. Tak jak w przypadku pozostałych zadań, proszę skutecznie przetestować działanie układu. Następnie proszę zbudować praktyczny układ, który za pomocą przełączników binarnych pozwoli ustawić żądaną czterobitową wartość, a następnie przy pomocy piątego przełącznika uruchomi szeregową transmisję odczytywanej wartości.
\subsection{Komentarz twórczy}
 Co tutaj?
\end{document}